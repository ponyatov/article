\secrel{Documentation centric sample application: this article
self-representation}\secdown 
\secrel{Documentation structure elements}
\secrel{Google Docs API interfacing}

\secrel{\LaTeX\ render subsystem}\label{dynatex}

Going ideas noted in this
\href{https://tex.stackexchange.com/questions/633/is-there-any-way-to-get-real-time-compilation-for-latex/391902#391902}{\so\
question}, let's realize \LaTeX\ render system able to work with local document modifications in real-time manner.

\secrel{Simple script for GraphViz graph extraction}

For purpose of simple visualization you can use
\href{https://github.com/ponyatov/article/raw/master/neo2viz.py}{neo2viz.py}
sample script I use for this paper figures.

Giving Cypher query from command line you will get .dot script on stdout can be
 redirected to any file.
 
Graph db elements should have some attributes with \verb|dot_| prefix
corresponds to graphviz attributes.
For node and edge labeling \verb|title:| attribute will be used.

Script require Python BOLT driver installation described in next section.

\secrel{\neo\ \py\ connection driver install}

To use \neo\ in \py\ you must follow this manual to install BOLT driver for
\py. Run this:

\begin{verbatim}
unix:~$ sudo pip install neo4j-driver
\end{verbatim}

and consult
\href{https://github.com/ponyatov/article/raw/master/neo2viz.py}{neo2viz.py}
source as sample for using \neo\ in \py.

\secup