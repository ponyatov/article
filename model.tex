\secrel{Graph Data Model}\secdown

For first time we will use attribute graph model. Nodes represents data
elements, and edges or links shows relationships between nodes. Nodes and edges
can have properties. \neo\ also includes special case of properties\ --- labels,
you can think about label as property without value, or has value equal to
itself.

So graph structure can represent wide range of different things, look at your
whiteboard with some schemes for example. Let’s look on our prototype system
architecture at \autoref{fig:architecture}. As you can see that there is no
reason to add any extra notes: all knowledge embodied in this graph is visually
understandable. As other example, see \autoref{fig:mobile} later.

\fig{tmp/architecture.png}{architecture}{Active graph system
architecture}{width=\textwidth}

\clearpage

\secrel{Graph Databases}\secdown

Most graph databases uses data models, not so powerful to implement graphs shown
on figures \autoref{fig:architecture},\autoref{fig:mobile} natively. As you can
see, there is some elements not supported: grouping and containers, links on
links, N-ary relations with \mbox{$N>2$} (and custom visualization
representations). Some of this can be described in hypergraph (see
\autoref{hyper}), but for currently existing graphdb engines we need some
transformations.

\secrel{\neo}

The most simple and easy install graph database is
\href{https://neo4j.com/}{\neo}. It has lot of data model limitations, in detail
described in \autoref{hyper}\ section, but we will use \neo\ as graph storage
for our prototype system, ready to use just now:

\begin{itemize}[leftmargin=*]
  
\item
install
\href{http://www.oracle.com/technetwork/java/javase/downloads/index.html}{\java\
SE} for your system\note{install full
\href{http://www.oracle.com/technetwork/java/javase/downloads/jdk8-downloads-2133151.html}{JDK}
if you not limited in disk space, or 
\href{http://www.oracle.com/technetwork/java/javase/downloads/jre8-downloads-2133155.html}{JRE}
only}.
Download .tag.gz package and
unpack it in user home directory, or somewhere else
\begin{verbatim}
user@unix$ cd ~ ; tar zx < Downloads/
\end{verbatim}

\item
add to \verb|JAVA_HOME| and \verb|PATH| environment variables points to your
\java\ install/bin directory

add to file \verb|~/.profile|
\begin{verbatim}
. ~/.setenv
\end{verbatim}
add file \verb|~/.xsessionrc|
\begin{verbatim}
. ~/.setenv
\end{verbatim}
add file \verb|~/.setenv|
\begin{verbatim}
export JAVA_HOME="/home/user/jdk1.8.0_131"
export PATH="$JAVA_HOME/bin:$PATH"
\end{verbatim}

\item
download \href{https://neo4j.com/download/community-edition/}{\neo\ community
edition} distribution archive and unpack it by commands:
\begin{verbatim}
unix:~$ tar zx < Downloads/neo4j-community-3.2.3-unix.tar.gz
\end{verbatim}
\begin{verbatim}
windows> unzip Downloads/neo4j-community-3.2.3-windows.zip
\end{verbatim}

\item 

start \neo\ server in console mode (recommended) from your home directory:
\begin{verbatim}
user@unix$ ~/neo4j-community-3.2.3/bin/neo4j console
\end{verbatim}
or in daemon mode under current user
\begin{verbatim}
user@unix$ ~/neo4j-community-3.2.3/bin/neo4j start
\end{verbatim}
You can stop \neo\ server by pressing \keys{Ctrl-C}\ in console or run stop
command in daemon mode
\begin{verbatim}
user@unix$ ~/neo4j-community-3.2.3/bin/neo4j stop
\end{verbatim}
If your Java was not installed properly, you will get errors like
\begin{verbatim}
ERROR: JAVA_HOME is incorrectly defined
as /home/ponyatov/jdk1.8.0_131
\end{verbatim}
(the executable \verb|/home/ponyatov/jdk1.8.0_131/bin/java| does not exist)

Also you can install only JRE and run \neo\ with directly selected Java runtime
directory:
\begin{verbatim}
user@unix$ JAVA_HOME=~/jre1.8.0_144 \
    ~/neo4j-community-3.2.3/bin/neo4j console
\end{verbatim}

\item connect to \url{http://localhost:7474/}\ with any JavaScript-enabled web
browser

\item connect to graphdb server via command
\begin{verbatim}
:server connect
host: bolt://localhost:7687
username: neo4j
password (default): neo4j
\end{verbatim}
\menu{Connect}

\item you will be asked for new password for \neo\ user
\begin{verbatim}
New password: ********
Repeat new password: ********
\end{verbatim}
\menu{Change password}
  
\end{itemize}
\secup

\secrel{Sample program graph model}

For demo purposes let’s see simple program for UNIX in pure \C{}:

\lst{src/hello.c}{hello.c}{Hello World}

\fig{tmp/hello.png}{hello}{Hello World simple program
in \C{}}{width=\textwidth}

Here we see types taxonomy for primitive subset of \C{}\ language, \verb|main()|
function with arguments and \verb|return| statement. Note that function
arguments in \C\ must be ordered, so we use \emph{next} relation to add this
order. Other notable thing is double return relation of main() function: this is
sample of inherited relation\ --- \verb|(main:function)-[:return]->(int:type)|
relation was inherited from first \verb|main-[:return]->(0:int)| relation
corresponds to \verb|return 0| statement. \textbf{ako} (a-kind-of, superclass)
and \textbf{isa} (is-a) relations corresponds to inheritance and instantiation.


\secup
