% article draft @ https://www.overleaf.com/read/kwxjwqhbtvfp
\documentclass[oneside,12pt]{article}
\usepackage[a5paper,margin=5mm]{geometry}%,showframe
%\usepackage{showframe}
\usepackage[utf8]{inputenc}
\usepackage{xcolor}
\usepackage[colorlinks,
	linkcolor={red!50!black},citecolor={blue!50!black},urlcolor={blue!80!black}]{hyperref}

%% cross references [hrefs]
\usepackage{caption}
\def\figureautorefname~#1\null{fig.#1}
\newcommand{\email}[1]{$<$\href{#1}{#1}$>$}

%% figures
\usepackage[pdftex]{graphicx}
\newcommand{\fig}[4]{
\begin{figure}[ht]
{\centering\noindent\includegraphics[#4]{#1}
\protect\caption{#3}\label{fig:#2}}
\end{figure}
}
\newcommand{\figref}[1]{$[$fig.{\ref{#1}}$]$}

% relative sectioning
\usepackage{ifthen}
\newcounter{secdepth}\setcounter{secdepth}{0}
\newcommand{\secup}{\addtocounter{secdepth}{1}}
\newcommand{\secdown}{\addtocounter{secdepth}{-1}}
\newcommand{\secrel}[1]{
\ifthenelse{\equal{\value{secdepth}}{0}}{\part{#1}}{}
\ifthenelse{\equal{\value{secdepth}}{-1}}{\chapter{#1}}{}
\ifthenelse{\equal{\value{secdepth}}{-2}}{\section{#1}}{}
\ifthenelse{\equal{\value{secdepth}}{-3}}{\subsection{#1}}{}
\ifthenelse{\value{secdepth} < -3}{\subsubsection{#1}}{}
}
\newcommand{\secly}[1]{\section*{#1}\addcontentsline{toc}{section}{#1}}
\newcommand{\subsecly}[1]{\section*{#1}\addcontentsline{toc}{subsection}{#1}}

% misc

\renewcommand{\emph}[1]{\textbf{#1}}
\newcommand{\note}[1]{\footnote{\ #1}}
\newcommand{\term}[1]{\textcolor{green}{#1}}

% [nosep] option in lists/enums
\usepackage{enumitem}
% frame box
\usepackage{framed}

%% languages/programs

\newcommand{\prog}[1]{\textit{#1}}
\newcommand{\lang}[1]{$#1$}

\newcommand{\C}[1]{\lang{C_{#1}}}
\newcommand{\cpp}{\lang{C_{+^+}}}
\newcommand{\java}{\lang{Java}}
\newcommand{\py}{\lang{Python}}

\newcommand{\prolog}{\lang{Prolog}}
\newcommand{\yp}{\lang{YieldProlog}}

\newcommand{\xsb}{\prog{XSB}}

\newcommand{\neo}{\prog{neo4j}}

\newcommand{\so}{$StackOverflow$}

%% listings
\usepackage{verbatim}
\usepackage{listings}
\lstset{
basicstyle=\small,
frame=single,
numbers=left,numberstyle=\small,numbersep=2mm,
tabsize=4,
keywordstyle=\textbf,
commentstyle=\color{blue}\textbf
}
\newcommand{\lst}[3]{\lstinputlisting[title=#2\ :\ #3]{#1}}
\newcommand{\lstx}[4]{\lstinputlisting[language=$#4,title=#2\ :\ #3]{#1}}

%% software menu & keys
\usepackage[os=win]{menukeys}
\usepackage{amssymb} % windows key
\newcommand{\winstart}{$\boxplus$}
\newcommand{\winr}{\keys{\winstart+R}}
\newcommand{\lms}{$\lhd$}
\newcommand{\dblms}{$\lhd\lhd$}
\newcommand{\rms}{$\rhd$}
\newcommand{\checkbox}{$\boxtimes$}
\newcommand{\uncheckbox}{$\square$}



\title{
Using active hypergraph database\\
as software development approach:\\
\Huge{Graph Driven Programming}}

\author{\small{\copyright\ Dmitry Ponyatov \email{dponyatov@gmail.com}, free researcher, 2017}}

\begin{document}
\maketitle
\begin{abstract}\noindent
Software development and program architecture engineering has grows complexity
in last decades. The existing zoo of programming languages, para\-digms,
hardware platforms and operating systems, combined with the software development
market requirements, exceeds the capabilities of a particular individual
developer. Thus, it is required to create a powerful tool for compensation of
software systems development complexity, that combines design tools, RAD, MDP,
simulation debugging, static and dynamic analysis of programs, parallel
computing support, and, first of all, \emph{an expert system and knowledge bases
provides knowledge storage, fuzzy search and inference} not only in software
engineering, but also in applied fields (mathematics and numerical methods,
physics, chemistry, mechanics and structural engineering, CAD, electronics,
digital signal processing, business planning, accounting and logistics, task and
time management, etc).
\end{abstract}

\tableofcontents\secdown\secdown

\secly{Introduction}

Software development and program architecture engineering has grows complexity
in last decades. This problem is especially notable in hard real-time control
systems applied in safety-critical domains: avionics, aerospace, automobiles,
and medicine, and covered by tons of safety reglaments and standards.
\cite{book1}

If we look on some random typical software developer vacancies, we see huge
amount of conventional requirements on knowledge and skills (emphasized in
text), making developer live damn scary: to make yourself promising in your
profession, you must spend an everyday self-learning in your work and spare
time.

\begin{framed}\noindent
An experienced \emph{Java} programmer with \emph{R} experience is sought to
participate in the continued design and development of a flexible, open-ended
platform for \emph{bioinformatics} data integration, \emph{analysis}, and
\emph{visualization}. This open-source platform \emph{distributes computations}
across client, server, and \emph{grid} components. The candidate is expected to
contribute to architecture design, \emph{visualization}, and \emph{web-GUI}
implementation, \emph{integration} of \emph{existing algorithms}/programs,
troubleshooting software bugs, and producing technical documentation. The
position requires a Bachelor's Degree in \emph{Computer Science},
\emph{Engineering}, \emph{Mathematics}, \emph{Physics}, \emph{Bioinformatics},
or in a similar field or equivalent in education and experience, plus a minimum
of three (3) years of related experience. At least 1 year of experience
\emph{programming in a UNIX environment} is required, as well as 1 year of
experience with \emph{R} and \emph{Java} programming, including experience with
distributed file systems and cluster compute management, shell scripting on UNIX
and experience in one scripting language, preferably \emph{Python}. Also
desirable is familiarity with \emph{database} driven projects using \emph{MySQL}
or a similar database system, \emph{SQL}, \emph{JDBC}, \emph{Linux}, and use of
appropriate IDEs (e.g., \emph{Eclipse}), project management tools (e.g.,
\emph{ant}, \emph{subversion}), web technologies (web service frontend triple
\emph{HTML}/\emph{CSS}/\emph{JavaScript}, \emph{tomcat}, \emph{php},
\emph{drupal}).
\end{framed}

\begin{framed}\noindent
Participate on requirement analysis, design, development and \emph{testing}.\\
Knowledge developing and debugging using \emph{C} (Kernel)/\emph{C++},
\emph{BASH}\\
Expert knowledge with \emph{algorithm and data structure} design \\
Strong knowledge and extensive \emph{Python}/\emph{UNIX}/Linux development skills \\
Knowledge of programming for computer networking IP protocol set, \emph{TCP/IP}, branch to interoffice connectivity, \emph{network security}, and \emph{encryption} is a must \\
Related Terms: C (\emph{Linux kernel development}), \emph{C++}, \emph{Python},
\emph{Perl}, Linux, UNIX, BASH, \emph{XML}, \emph{TCP/IP} and \emph{SSL}\\
Participate on requirement analysis, design and development on \emph{Java}, \emph{J2EE}, \emph{C/C++}. Develop applications, work on \emph{Amazon Web Services}, work on integrated development environments. Requires B.S. degree in Computer Science or related area plus 5 years of experience in software development, or, alternatively, M.S. degree in Computer Science or related area. Knowledge of Java, J2EE, C/C++, \emph{JSF}, \emph{Servlets}, \emph{JSP}, \emph{Java Web Services}, \emph{XML}, \emph{Ajax}, \emph{jQuery}, \emph{HTML} and \emph{CSS}; familiarity with technologies like \emph{JDBC}, \emph{SOAP} and \emph{OOD}; knowledge of \emph{PL/SQL} Procedures, Functions and Triggers for \emph{Oracle}, \emph{MySQL} and \emph{SQL Server} databases; ability to work on \emph{Linux Kernel} module, \emph{TCP/IP stack} application and \emph{network protocols} implementation; familiarity with \emph{XML} documents and use of \emph{DTD}, \emph{SCHEMA} and parsing using \emph{SAX}, \emph{DOM} and transformations using \emph{XSL}, \emph{XSLT}, and \emph{XPATH}.
\end{framed}

\begin{framed}\noindent
\begin{itemize}[nosep,leftmargin=*]
\item Design, develop, and release sophisticated Windows GUI applications in an iterative environment using \emph{C\#} \emph{.NET}, \emph{Windows Forms}, and the \emph{Win32 API}
\item At least 2 years Windows GUI programming (C\# .NET, Windows Forms, \emph{WPF}, Win32 API)
\item \emph{JavaScript} single page application development (AngularJS, Knockout.JS or similar)
\item Familiarity with \emph{SQL} and \emph{databases}
\item Experience with DevExpress Windows Forms/WPF  UI Framework
\item Experience with \emph{financial software} and/or financial trading, especially in fixed income, a plus
\item Experience with \emph{real-time systems} and rapid iterations a plus
\item Experience with \emph{Multithreading} and Interfacing with \emph{TIBCO}
\item Knowledge of \emph{VB}, \emph{Python}, \emph{C++}
\item Experience with \emph{Subversion} Source Control
\item Experience with \emph{Unit Testing}, Automated UI Testing, and Code Coverage tools.
\item Knowledge of S.O.L.I.D principles, Requirement Analysis, and QA Test Management
\item Experience using tools like ReSharper, Beyond Compare,  Tortoise
\end{itemize}
\end{framed}

\begin{framed}\noindent
\begin{itemize}[nosep,leftmargin=*]
\item \emph{C programming} experience
\item Software Development using '\emph{Internet of Things}' standards, including \emph{OCF}
\item Software Development using IP based \emph{internet protocols} including \emph{SSDP}, \emph{UDP}, \emph{TCP}, \emph{TLS}, CoAP, \emph{HTTP}, \emph{MQTT}, Thread, \emph{WiFi}, WebSockets
\item Development on \emph{Windows} (\emph{Win32 API}, \emph{.NET}) and Linux/\emph{Unix} Platforms
\item Managing microservices in \emph{AWS}
\item \emph{Testing} and debugging \emph{embedded electronics}
\item Rapid prototyping using available tools such as: \emph{Arduino}, \emph{Raspberry} Pi
\item Experience and continued interest in learning and \emph{deploying} unfamiliar languages, \emph{frameworks}, and other tools
\item \emph{Python} or \emph{Shell} Scripting language familiarity
\item Eager to learn and adapt to new and \emph{changing requirements}
\item Bachelor's degree in ECE, EE, CS or related field.
\item \emph{AWS} experience or other \emph{cloud providers}
\end{itemize}
\end{framed}

My previous job was linked with hardware and firmware design. So my thoughts was
about it will be great to have some \prog{Wolfram Alpha} like expert system with
lot of engineering knowledge databases (basic three sciences
Math/Physics/Chemistry, Electronics, Computer Science and Programming,
especially software synthesis using Model Driven programming, Digital Methods
and DSP), which can help me in my work.

First I dig into pure \prolog\ and read lot of books on expert systems design,
and in result it was no progress at all. In my first senses, \prolog\ is only
able to implement knowledge bases about Socrates siblings.

So in last year I faced with graph databases, especially \neo, and it looks
great for knowledge representation, and processing via graph rewrite. I feel
that graphs can be natively represented in form of \prolog\ (especially
\lang{\href{https://en.wikipedia.org/wiki/HiLog}{HiLog}}) clauses (edge as
predicate, and nodes as arguments), so for inference purposes it looks able to
export graph into some modern \prolog\ system like
\prog{\href{http://flora.sourceforge.net/}{Ergo/Flora}} or \xsb, do inference,
and push back results via graph rewrite requests. Or inference can be done at
graphdb engine level using some custom plugin.

My interest on hypergraph linked with \neo\ limitations of representing not more
than binary relations, so some hypothetical clause like
\\\verb|family(father,mother,sister,brother,cat)|\\ require lot of underground
work with 2-ary primitives in \neo\ data model.

\secrel{Graph Data Model}\secdown

For first time we will use attribute graph model. Nodes represents data
elements, and edges or links shows relationships between nodes. Nodes and edges
can have properties. \neo\ also includes special case of properties\ --- labels,
you can think about label as property without value, or has value equal to
itself.

So graph structure can represent wide range of different things, look at your
whiteboard with some schemes for example. Let’s look on our prototype system
architecture at \autoref{fig:architecture}. As you can see that there is no
reason to add any extra notes: all knowledge embodied in this graph is visually
understandable. As other example, see \autoref{fig:mobile} later.

\fig{fig/architecture.png}{architecture}{Active graph system
architecture}{width=\textwidth}

\clearpage

\secrel{Graph Databases}\secdown

Most graph databases uses data models, not so powerful to implement graphs shown
on figures \autoref{fig:architecture},\autoref{fig:mobile} natively. As you can
see, there is some elements not supported: grouping and containers, links on
links, N-ary relations with \mbox{$N>2$} (and custom visualization
representations). Some of this can be described in hypergraph (see
\autoref{hyper}), but for currently existing graphdb engines we need some
transformations.

\secrel{\neo}

The most simple and easy install graph database is
\href{https://neo4j.com/}{\neo}. It has lot of data model limitations, in detail
described in \autoref{hyper}\ section, but we will use \neo\ as graph storage
for our prototype system, ready to use just now:

\begin{itemize}[leftmargin=*]
  
\item
install
\href{http://www.oracle.com/technetwork/java/javase/downloads/index.html}{\java\
SE} for your system\note{install full
\href{http://www.oracle.com/technetwork/java/javase/downloads/jdk8-downloads-2133151.html}{JDK}
if you not limited in disk space, or 
\href{http://www.oracle.com/technetwork/java/javase/downloads/jre8-downloads-2133155.html}{JRE}
only}.
Download .tag.gz package and
unpack it in user home directory, or somewhere else
\begin{verbatim}
user@unix$ cd ~ ; tar zx < Downloads/
\end{verbatim}

\item
add to \verb|JAVA_HOME| and \verb|PATH| environment variables points to your
\java\ install/bin directory

add to file \verb|~/.profile|
\begin{verbatim}
. ~/.setenv
\end{verbatim}
add file \verb|~/.xsessionrc|
\begin{verbatim}
. ~/.setenv
\end{verbatim}
add file \verb|~/.setenv|
\begin{verbatim}
export JAVA_HOME="/home/user/jdk1.8.0_131"
export PATH="$JAVA_HOME/bin:$PATH"
\end{verbatim}

\item
download \href{https://neo4j.com/download/community-edition/}{\neo\ community
edition} distribution archive and unpack it by commands:
\begin{verbatim}
unix:~$ tar zx < Downloads/neo4j-community-3.2.3-unix.tar.gz
\end{verbatim}
\begin{verbatim}
windows> unzip Downloads/neo4j-community-3.2.3-windows.zip
\end{verbatim}

\item 

start \neo\ server in console mode (recommended) from your home directory:
\begin{verbatim}
user@unix$ ~/neo4j-community-3.2.3/bin/neo4j console
\end{verbatim}
or in daemon mode under current user
\begin{verbatim}
user@unix$ ~/neo4j-community-3.2.3/bin/neo4j start
\end{verbatim}
You can stop \neo\ server by pressing \keys{Ctrl-C}\ in console or run stop
command in daemon mode
\begin{verbatim}
user@unix$ ~/neo4j-community-3.2.3/bin/neo4j stop
\end{verbatim}
If your Java was not installed properly, you will get errors like
\begin{verbatim}
ERROR: JAVA_HOME is incorrectly defined
as /home/ponyatov/jdk1.8.0_131
\end{verbatim}
(the executable \verb|/home/ponyatov/jdk1.8.0_131/bin/java| does not exist)

Also you can install only JRE and run \neo\ with directly selected Java runtime
directory:
\begin{verbatim}
user@unix$ JAVA_HOME=~/jre1.8.0_144 \
    ~/neo4j-community-3.2.3/bin/neo4j console
\end{verbatim}

\item connect to \url{http://localhost:7474/}\ with any JavaScript-enabled web
browser

\item connect to graphdb server via command
\begin{verbatim}
:server connect
host: bolt://localhost:7687
username: neo4j
password (default): neo4j
\end{verbatim}
\menu{Connect}

\item you will be asked for new password for \neo\ user
\begin{verbatim}
New password: ********
Repeat new password: ********
\end{verbatim}
\menu{Change password}
  
\end{itemize}
\secup

\secrel{Sample program graph model}

For demo purposes let’s see simple program for UNIX in pure \C{}:

\lst{src/hello.c}{hello.c}{Hello World}

\fig{tmp/hello.png}{hello}{Hello World simple program
in \C{}}{width=\textwidth}

Here we see types taxonomy for primitive subset of \C{}\ language, \verb|main()|
function with arguments and \verb|return| statement. Note that function
arguments in \C\ must be ordered, so we use \emph{next} relation to add this
order. Other notable thing is double return relation of main() function: this is
sample of inherited relation\ --- \verb|(main:function)-[:return]->(int:type)|
relation was inherited from first \verb|main-[:return]->(0:int)| relation
corresponds to \verb|return 0| statement. \textbf{ako} (a-kind-of, superclass)
and \textbf{isa} (is-a) relations corresponds to inheritance and instantiation.


\secup

\clearpage
\secrel{Inference over graph}\secdown

The most complex and powerful thing is automated inference over graph
database\note{\href{https://stackoverflow.com/questions/44678792/how-to-reason-or-make-inferences-in-neo4j}{same
question} on \so}\ using set of forward inference trigger-like scripts,
implements rewrite rules, mixed with yield method \ref{yield}.

\lst{src/socrat.pl}{socrat.pl}{Classical Socrates sample in \prolog}

\fig{tmp/socrat.pdf}{fig:socrat}{Socrate's siblings}{width=0.75\textwidth}

\clearpage

\lst{src/socrat.log}{socrat.log}{SWI Prolog run log}

\secrel{\yp\ traversal over graph database}\label{yield}

In this section we'll see \href{http://yieldprolog.sourceforge.net/}{\yp}
techinique in details.

\begin{verbatim}
human(socrates).
human(aristotle).
human(plato).
\end{verbatim}

Multiple facts with same predicate must be groupped into one generator function: 

\lst{src/socrat1.py}{socrat.py}{facts in \yp}

\fig{tmp/socrat1.pdf}{fig:socrat1}{Can be seen as this graph
representation}{width=\textwidth}

\clearpage

We can pass an object into the iterator which it will use to return the value.
This is first step to \emph{\term{unification}: substituting variables in
inference process}.

\lst{src/socrat2.py}{socrat.py}{facts in \yp\ using \emph{variable}}

Next step we moving to \term{unification}. See
\href{http://www.amzi.com/articles/prolog_under_the_hood.htm}{here} and
\href{http://www.learnprolognow.org/lpnpage.php?pagetype=html&pageid=lpn-htmlse5}{here} 
how unification works, \emph{it is required to understand next code section}.

\lst{src/socrat3.py}{socrat.py}{Variables \term{unification}}


\secrel{Graph rewrite}

Rewrite method is most intuitive\ ---  we use \emph{match/update} template:
\emph{find} part of graph we interested in, and then \emph{update} selected
elements.


\secup

\secrel{HyperGraph generalization}\label{hyper}\secdown

Lot of generic graph data model limitations tends to use more sophisticated
structure\ --- hypergraph. Hypergraph generalizes element types\ --- nodes and
links becomes unificated elements called nodes (entities or vertices), which can
define relations between any number of other nodes. So, comparing to graph,
hypergraph is free to represent links on links, and any N-ary relations between
data elements.For practical reasons it is profitable to use hypergraph entities
with properties and labels (think about it as property without value).

\secrel{HyperGraph Databases}\secdown

\secrel{opencog}

\secrel{HGDB}

\secrel{Grakn.AI}

Grakn is power hypergraph engine specially designed for AI applications.\\
Go to \href{https://grakn.ai}{official cite} and follow its manual for system
installation\note{MacOS/Linux only}.\\
It has impressive look\&feel as the same impressive system size and computing
resources requirements: my not so old workstation crouches on server start with
empty database.

\secup

\secup
\secrel{(Hyper)Graph representation of typical computer science
structures}\secdown
\secrel{Containers}
\secrel{Attributed Abstract Syntax Tree}
\secrel{View/Controller in-graph modeling for dynamic web GUI}\secdown
\secrel{Custom styles in graph visualization}
\secrel{Widgets}
\secrel{Automated client code generation}
\secup
\secrel{Data Flow on triggers: active graph and event processing}
\secup

\secrel{Automated code generation (autogen © technology)}\secdown
\secrel{Conceptual programming\\(multi paradigm, generic and model-driven
programming)} 
\secrel{Data Flow automated parallelism}
\secrel{AST-based specialization \cite{book2}}
\secrel{Optimization on intermediate representations}
\secrel{Code Generation on high-level languages}
\secup
\secrel{Knowledge Databases and Expert systems in Software Development
domain}\secdown
\secrel{Knowledge representation}
\secrel{Fuzzy search}
\secrel{Frame logic Inference engine \cite{book3}}\secdown
\secrel{Ergo/Flora interfacing}

Well-packages Frame-inference system you can download @ flora.sourceforge.net -
it is Flora-2, free version of commercial system Ergo.
\secup
\secrel{Software Development Knowledge bases collection}\secdown
\secrel{Algorithms and data structures KB}
\secrel{Programming languages \& frameworks KB}
\secrel{Software Architecture KB}
\secrel{Distributed Computing KB}
\secup
\secup

\secrel{Demo application: personal task/time planning system}\secdown

\secrel{Distributed architecture with mobile Android client side}

You can download source code and release binary build .apk from

\bigskip
\url{https://github.com/ponyatov/hedge}
\bigskip

\fig{tmp/mobile.png}{mobile}{Mobile client implementation}{width=0.95\textwidth}
\clearpage

Current client version still does not support graph-driven dynamic GUI, but can
be used for sensor data collection for experimenting with \neo\ scripts and
planning and analysis algorithms.

For client use you must create \neo\ database instance on external server
\note{\href{http://www.graphenedb.com/}{graphenedb.com} can be used with free
account}, and put BOLT connection line into app configuration. Current version
have problems with on-device detached mirror\note{\neo\ for Android required},
so you can use application only in online mode. Online connection to remote
graphdb was done via \neo\ JDBC by using
\href{https://github.com/neo4j-contrib/neo4j-jdbc#minimum-viable-snippet}{this
manual}.

\secrel{What is personal planning}

In everyday life, I constantly find myself failing and forgetting to do a lot of
things. This is not because of the laziness or folly, but because of the many
competing interests that are characteristic of most of people. We have axes in
money thru job, hobby as fallback for achievement self-satisfaction in
professional, activities for health care, obligations to family and friends,
some of us have new interests and thirst for knowledge. So to achieve success we
can rely on the achievements of modern electronics, and make our T-800 CPU from
an Android mobile phone and inference system based on neo4j graph database.

\bigskip
Go to \href{https://www.graphenedb.com/}{graphenedb.com} or other neo4j online
hosting, sign up/login and 
\begin{itemize}[nosep,leftmargin=*]
  \item 
create new Planning database (free hobby sandbox is good for first time):\\
\emph{don't forget to select modern version neo4j 3.2.3},\\
it has some goodies in Cypher syntax and webgui.
  \item 
Create user for remote access via BOLT protocol: \verb|hedge|.
  \item 
Save your password and BOLT access line:\\ 
\verb|bolt://hobby-bahdmkgcjildgbkepggmibpl.dbs.graphenedb.com:24786|
  \item 
Switch to web GUI via Databases/Planning/Overview/Neo4j Browser/Launch
\end{itemize}
\bigskip

\noindent
For doing some scripting you can use samples hosted at\\
\url{https://github.com/ponyatov/article} :

\bigskip
\begin{tabular}{l l}
\verb|d3view.html + lib/|
&
neo4j database visualizer in JavaScript/D3\\&(run make to download offline
libs)
\\
\verb|neo2viz.py| & \py\ client makes GraphViz plots \\
\end{tabular}
\bigskip


First create yourself as center of Universe:
\begin{verbatim}
$ merge
    (:me{ firstname:'Dmitry', secondname:'Ponyatov',
          email:'dponyatov@gmail.com'},
          title:'Ponyatov')
-[:isa]->
    (:class:Person{title:'Person'})
\end{verbatim}

\fig{tmp/person1.png}{person1}{You is-a Person}{width=0.5\textwidth}

Not so impressive, so we need some good look for our data schemes. We can do it
using some View\note{I mean Model View Controller template}\ attributes, which
will set view representation.

So, first make query\note{later we will make it triggerable} which will update
all class-looking elements:

\begin{verbatim}
$ match (x)-[r:isa]->(y) set
        r.color='blue',r.fontcolor='blue',
        y:class,y.shape='box',y.fillcolor='lightpink'
    return x,r,y 
\end{verbatim}

And emphasize youself as center element:

\begin{verbatim}
$ match (x:me) set x.fillcolor='green',x.shape='circle'
\end{verbatim}

\fig{tmp/person2.png}{person2}{Colorized variant looks
much better}{width=0.5\textwidth}

We use \href{http://graphviz.org/}{GraphViz} so available \emph{view
attributes}\ --- shape and color names see in manual:
\bigskip

\begin{tabular}{l l}
Colors:&\url{http://www.graphviz.org/doc/info/colors.html}\\
Shapes:&\url{http://www.graphviz.org/doc/info/shapes.html}\\
\end{tabular}


\secrel{Android KB}
\secrel{Resource analysis in task management and logistics applications}
\secrel{Geopositioning in logistic applications}

\secup

\secrel{Documentation centric sample application: this article
self-representation}\secdown 
\secrel{Documentation structure elements}
\secrel{Google Docs API interfacing}

\secrel{\LaTeX\ render subsystem}\label{dynatex}

Going ideas noted in this \so\ question, let's realize \LaTeX\ render system
able to work with local document modifications in real-time manner.

\secrel{Simple script for GraphViz graph extraction}

For purpose of simple visualization you can use
\href{https://github.com/ponyatov/article/raw/master/neo2viz.py}{neo2viz.py}
sample script I use for this paper figures.

Giving Cypher query from command line you will get .dot script on stdout can be
 redirected to any file.
 
Graph db elements should have some attributes with \verb|dot_| prefix
corresponds to graphviz attributes.
For node and edge labeling \verb|title:| attribute will be used.

Script require Python BOLT driver installation described in next section.

\secrel{\neo\ \py\ connection driver install}

To use \neo\ in \py\ you must follow this manual to install BOLT driver for
\py. Run this:

\begin{verbatim}
unix:~$ sudo pip install neo4j-driver
\end{verbatim}

and consult
\href{https://github.com/ponyatov/article/raw/master/neo2viz.py}{neo2viz.py}
source as sample for using \neo\ in \py.

\secup
\secrel{Large-scale dynamic object system}\secdown

I’m impressed with SmallTalk language and interactive system see
\cite{bluebook}, but I don’t like SmallTalk syntax itself. I’m joined to
StackOverflow
\href{https://stackoverflow.com/questions/5732017/python-development-environments-like-smalltalk/42111909}{question}
about SmallTalk in Python, and I’m dreaming about :

\begin{itemize}[nosep,leftmargin=*]
  \item 
system with live objects (dynamic object system) distributed over cluster 
  \item 
dedicated servers (beowulf cluster) or the best is cloud of user computers
  \item 
Pythonic control and programming language syntax able to work on 
  \item 
all modern computer platforms (PC, embedded and mobile devices) with 
  \item 
widely
used OSes (Windows, MacOS, Linux, Android) as hosted OS.
\end{itemize}

\secrel{Interpreter}

System user interface based on command line using tiny interpreter:
\url{https://github.com/ponyatov/I}

This interpreter also will be demo application for this paper showing some hints
on GDP methodology.

So I’m going to use this interpreter as Python replacement for scripting in my
\href{https://github.com/ponyatov/L}{embedded Linux} projects: Python has
problematic build process require lot of side fixes (see BuildRoot sources), so tiny clean C++ interpreter core will be
much more sufficient, especially for
\href{https://forum.arduino.cc/index.php?topic=442285.msg3390896#msg3390896}{non-Linux
hardware devices} made on modern Cortex-M microcontrollers.

\secrel{Object model with message passing}

\secdown


\secrel{Hardware project sample: Hobby-grade PLC}\secdown

\url{https://github.com/ponyatov/HobbyPLC}
\bigskip

This section describes GDP from viewpoint of hardware design: one of my interest
is numerical control systems, CAD design and applications, and manual and
\href{https://en.wikipedia.org/wiki/Digital_manufacturing}{digital
manufacturing}, so here I will show some elements of applying GDP in
non-programmatic area.
\secup

\addcontentsline{toc}{section}{List of Figures}
\listoffigures

\secly{Acknowledgements}

We thank Richard Stallman, Linus Torvalds, and all GNU/Linux community for lot
of software available for free and beer. We are grateful to the anonymous
referees for their comments, making this paper better and better. Thanks a lot
to \neo\ team for its grateful graph database server with incredible easy
install and run.

\begin{thebibliography}{99}
\addcontentsline{toc}{section}{References}

\bibitem{book1}
\href{https://insights.sei.cmu.edu/sei_blog/2015/09/managing-software-complexity-in-models.html}{\textbf{Managing
Software Complexity in Models}}\\

\bibitem{book2}
\href{https://www.repository.cam.ac.uk/bitstream/handle/1810/247929/Urma%20and%20Mycroft%202013%20Science%20of%20Computer%20Programming.pdf}{\textbf{Source-Code
Queries with Graph Databases\ --- with Application to Programming Language Usage and Evolution}}\\
\textit{Raoul-Gabriel Urma, Alan Mycroft}\\
Computer Laboratory, University of Cambridge

\bibitem{book3}
\href{http://www3.cs.stonybrook.edu/~kifer/TechReports/flogic.pdf}{\textbf{Logical
Foundations of Object-Oriented and Frame-Based Languages}}\\
\textit{Michael Kifer, Georg Lausen, James Wu}

\bibitem{minsky}
\href{https://web.media.mit.edu/~minsky/papers/Frames/frames.html}{\textbf{A
Framework for Representing Knowledge}}\\
\textit{Marvin Minsky}\\
MIT-AI Laboratory Memo 306, June, 1974.

\bibitem{bluebook} 
\href{http://stephane.ducasse.free.fr/FreeBooks/BlueBook/Bluebook.pdf}{\textbf{Smalltalk-80:
The Language and its Implementation}}\\
\textit{Adele Goldberg , DavidRobson}\\
Xerox Palo Alto Research Center\\
Addison Wesley, 1983.\\
ISBN 0-201-11371-6.\\

\bibitem{Rodriguez}
\href{https://markorodriguez.com/2011/02/23/knowledge-representation-and-reasoning-with-graph-databases/}{Knowledge
Representation and Reasoning with Graph Databases}\\
\textit{Marko A. Rodriguez}

\bibitem{Gardner}
\textbf{\href{http://www.cs.cmu.edu/~mg1/thesis.pdf}{Reading and Reasoning
with Knowledge Graphs}}\\
\textit{Matthew Gardner}

\end{thebibliography}

\end{document}
